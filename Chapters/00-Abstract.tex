\thispagestyle{plain} % Page style without header and footer
% \pdfbookmark[1]{Resumo}{resumo} % Add entry to PDF
% \chapter*{Resumo} % Chapter* to appear without numeration
% \blindtext

% \keywordspt{Keyword A, Keyword B, Keyword C.}

% \blankpage

\pdfbookmark[1]{Abstract}{abstract} % Add entry to PDF
\chapter*{Abstract} % Chapter* to appear without numeration

\acrfull{piv} is a technique that tracks particles within the flow of fluid around an airfoil to map a flow field. This experiment makes use of an airfoil in a wind tunnel, a laser illuminating smoke particles, and a high speed camera that captures the motion of the particles when illuminated by the laser. By taking pictures in extremely short succession, we determined instantaneous velocity vectors for each particle. Then, by analyzing hundreds of frames taken in short succession, we created averaged velocity fields that more accurately represent the flow. This process allowed us to analyze the wake characteristics—like the shape, vorticity, and turbulent kinetic energy. The \acrshort{piv} measurements we analyzed in this lab correctly showed the flow characteristics of an airfoil and generally matched our previous and expected results.