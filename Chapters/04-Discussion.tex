\chapter{Discussion} \label{cp:discussion}

The flow characteristics and uncertainties are different for instantaneous \acrshort{piv} measurements as opposed to the averaged, or ensemble, measurements. The uncertainty in the velocity field is inversely related to the number of snapshots. More precisely, as the number of instantaneous snapshots increases, the uncertainty decreases. Generating ensemble data smooths out fluctuations and noise in the data, and provides a more representative capture of the flow field's behavior. To show this, examine the difference between \autoref{fig:instant_aoa8} and \autoref{fig:averaged_piv_aoa8}, the first figure representing an instantaneous velocity and vorticity field and the latter being an averaged velocity and vorticity field for the same \acrshort{aoa}.

First, since the instantaneous velocity measurement has a different background color, it is immediately clear that the scale of vorticity is different. In this example, the ensemble data has a \textit{lower} max vorticity and a \textit{higher} minimum vorticity. This is consistent with the hypothesis that the ensemble data will smooth out outliers and momentary fluctuations in the flow—which may not be representative.

Second, note the flow region at approximately $x=-75$ and $y=-200$. In the instantaneous measurement, this area has extremely positive and negative vorticity occurring, perhaps indicating some type of vortex shedding. In the ensemble data, this same region also shows positive and negative vorticity in approximately the same shape, but the absolute difference between the positive and negative vorticity is \qty{25}{\percent} the absolute difference of the instantaneous measurement. This demonstrates an important idea: although ensemble data provides a more uniform and consistent characterization of the flow, small nuances in the flow may only be detected by observing instantaneous measurements.

Lastly, since the ensemble data is averaging a number of snapshots, the shape of the ensemble data is much more defined and clear than in the instantaneous measurement. This is particularly helpful in identifying the size and shape of the wake regions.

As shown in \autoref{fig:instant_aoa8} and \autoref{fig:instant_aoa16}, there are some regions with high velocity fluctuations or turbulent activity. These regions can be identified by finding local maxima and minima in the instantaneous measurements and comparing those locations to the averaged data. Turbulent flow regions exhibit greater velocity and vorticity variance compared to laminar flows; therefore, turbulent regions like those of wakes, shear layers, or towards the trailing edge at higher \acrshort{aoa}s will have higher uncertainty, especially if only observing the instantaneous measurement.

We were unable to compare the \acrshort{piv} wake measurements to the measurements made in the hotwire lab, due to erroneous data collection in the hotwire experiment. We were, however, able to compare our \acrshort{piv} wake measurement results obtained from the pitot rake measurements in lab six. Our conclusion from the pitot rake data was that as the \acrshort{aoa} increases, the wake region becomes wider and more turbulent. Examining \autoref{fig:averaged_half_chord}, this seems to match the \acrshort{piv} measurements taken in this lab. As the angle of attack increases, the wake region generally gets wider and more turbulent—as indicated by the higher velocity magnitudes in the wake region.

It should be noted that \autoref{fig:averaged_half_chord_aoa4} and \autoref{fig:averaged_half_chord_aoa8} look different from our pitot rake lab data and different from our theoretical expectations. We think it's less likely our wind tunnel was defying physics and more likely that we chose an improper point to sample the 