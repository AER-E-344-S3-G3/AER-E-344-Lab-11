\chapter{Conclusion}
\label{cp:conclusion}
To analyze the aerodynamic characteristics of an airfoil at different angles of attack, a \acrfull{piv} system was used to gather a 2D velocity field around an airfoil. The averaged velocity field for each angle of attack was used to find the vorticity distribution and turbulent kinetic energy around the airfoil. A wake profile was constructed by plotting the magnitude of the velocity as a function the \gls{y}-axis for a given \gls{x} position one half chord downstream of the trailing edge. The wake profile at each angle of attack shows a similar contour to other means of analysis, such as using a pitot rake or hot wire behind the airfoil. As the angle of attack increases, the vorticity, turbulent kinetic energy, and wake width also increase.