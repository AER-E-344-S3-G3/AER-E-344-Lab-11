\chapter{Conclusion}
\label{cp:conclusion}
To analyze the aerodynamic characteristics of an airfoil at different angles of attack, a \acrfull{piv} system was used to gather a 2d grid of data on the velocities across the test area. The velocity data averaged across all frames of an angle of attack is used to find the vorticity distribution and turbulent kinetic energy around the airfoil. The wake profile is seen by taking the velocity data across all $y$ positions for the $x$ position 1/2 chord downstream from the trailing edge. The wake profile at each angle of attack shows a similar contour to other means of analysis, such as using a pitot rake or hot wire behind the airfoil. As the angle of attack increases; the vorticity distribution, turbulent kinetic energy distribution, and wake profile also increase. 

From the results from the \acrshort{piv} system, we can conclude that it is very effective at giving a very detailed picture and accurate picture of what is happening in the flow field. With these results, we can have a better understanding of what is happening when there are different features in the flow, such as turbulence boundaries, boundary layer transitions, and vorticity.