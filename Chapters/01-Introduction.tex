\chapter{Introduction}
\label{cp:introduction}
Particle Image Velocimetry (PIV) is a powerful technique used in fluid mechanics to visualize and analyze flow fields. It provides information about fluid flow velocities by tracking the movement of particles in an area. PIV has applications in many industries, including aerospace, automobile, environmental engineering, and manufacturing, where understanding fluid behavior is critical for optimizing and designing.

The basic principle of PIV is to illuminate the flow field with a laser sheet or a pulsing light and capture the movement of tracer particles with a high-speed camera. A syncronizer is used to control the timing of the laser illumination and the camera acquisition. The displacement of particles is then calculated by comparing consecutive frames using a PIV software. Tracers should be neutrally buoyant and small enough to follow the flow perfectly but big enough to scatter the illumination lights efficiently.

This experiment aims to visualize and analyze the flow around a GA(W)-1 airfoil in a close-circuit low-speed wind tunnel. 
