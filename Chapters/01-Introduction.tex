\chapter{Introduction}
\label{cp:introduction}
Particle Image Velocimetry (PIV) stands out as a robust technique within fluid mechanics, serving to visualize and analyze flow fields precisely. Its applications span across diverse industries such as aerospace, automotive, environmental engineering, and manufacturing, where a deep understanding of fluid dynamics is crucial for optimization and design.

The fundamental principle underlying PIV involves illuminating the flow field either using a laser sheet or a pulsing light and then tracking the movement of tracer particles through high-speed camera imaging. A syncronizer coordinates the timing of the laser illumination and the camera acquisition. Subsequently, a PIV software calculates the displacement of particles by comparing consecutive frames. The tracers must be neutrally buoyant, small enough to follow the flow perfectly but big enough to scatter the illumination lights efficiently. If the tracer particles are too big, the gravitational force has a more significant influence on the movement of tracer particles, creating a source of error. 

This experiment aims to visualize and analyze the flow around a GA(W)-1 airfoil in a close-circuit low-speed wind tunnel at different angles of attack. The test section has a 12 x 12 cross-section area where the airfoil will be incorporated. The PIV system will measure the velocity fields of the tracer particles along the airfoil. For this experiment, we are using 1~55 um oil droplets. The Nd:YAG laser (NewWave Gemini 200) will illuminate using a system of optics and lenses while a high-resolution 12-bit CCD camera captures the images. Both the  CCD camera and the Nd:YAG laser are connected to a computer via a Digital Delay Generator (Berkeley Nucleonics, Model 565), which coordinates the timing between the laser and the camera. 


