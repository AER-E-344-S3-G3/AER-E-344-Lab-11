\chapter{Introduction}
\label{cp:introduction}
\acrfull{piv} is a robust technique within fluid mechanics, serving to visualize and analyze flow fields precisely. Its applications span across diverse industries such as aerospace, automotive, environmental engineering, and manufacturing, where a deep understanding of fluid dynamics is crucial for optimization and design.

The fundamental principle of \acrshort{piv} involves illuminating the flow field either with a laser sheet or a pulsing light and then tracking the movement of tracer particles through high-speed camera imaging. A synchronizer coordinates the timing of the laser illumination and the camera acquisition. After taking the images, a \acrshort{piv} software calculates the displacement of particles by comparing consecutive frames. The tracers must be neutrally buoyant, small enough to follow the flow perfectly but big enough to scatter the illumination lights efficiently. If the tracer particles are too big, the gravitational force has a more significant influence on the movement of tracer particles, creating a source of error \citep{hu2024}.

The goal of this experiment is to visualize and analyze the flow around a GA(W)-1 airfoil in a close-circuit, low-speed wind tunnel at different \acrshort{aoa}. The \acrshort{piv} system will measure the velocity of the tracer particles along the airfoil. For this experiment, we are using oil droplets that are from \qtyrange{1}{5}{\micro\meter} in diameter. The Nd:YAG laser (NewWave Gemini 200) illuminates using a system of optics and lenses while a high-resolution, high-speed camera captures the images. Both the camera and the laser are connected to a computer via a delay generator, which coordinates the timing between the laser and the camera \citep{lab11-manual}.